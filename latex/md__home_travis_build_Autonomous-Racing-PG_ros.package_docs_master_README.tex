Autonomous Racing -\/ Project Group -\/ TU Dortmund

\href{https://travis-ci.com/Autonomous-Racing-PG/ros.package}{\tt }

\subsection*{Getting Started}

These instructions will get you a copy of the project up and running

\#\#\# Install missing system dependencies 
\begin{DoxyCode}
1 sudo apt install libsdl2-dev
2 pip install torch
3 
4 # RangeLibc
5 sudo pip uninstall pip && sudo apt install python-pip
6 pip install cython
7 git clone http://github.com/kctess5/range\_libc
8 cd range\_libc/pywrapper
9 # Either:
10 ./compile.sh            # on VM
11 # Or:
12 ./compile\_with\_cuda.sh  # on car - compiles GPU ray casting methods
\end{DoxyCode}


\subsubsection*{Clone the Project}


\begin{DoxyCode}
1 git clone --recurse-submodules https://github.com/Autonomous-Racing-PG/ros.package.git arpg
2 cd arpg
\end{DoxyCode}


\subsubsection*{Move to R\+OS Workspace}


\begin{DoxyCode}
1 cd ros\_ws
\end{DoxyCode}


\#\#\# Install missing R\+OS dependencies 
\begin{DoxyCode}
1 rosdep install -y --from-paths src --ignore-src --rosdistro $\{ROS\_DISTRO\}
\end{DoxyCode}


\subsubsection*{Build R\+OS packages}


\begin{DoxyCode}
1 catkin\_make
\end{DoxyCode}


\subsubsection*{Run routines}


\begin{DoxyCode}
1 source devel/setup.bash (or setup.zsh)
\end{DoxyCode}


Now several routines can be started by executing the launch-\/files inside the {\bfseries launch/} directory. E.\+g.


\begin{DoxyCode}
1 roslaunch launch/gazebo\_car-teleop.launch
\end{DoxyCode}


\subsubsection*{Run tests}


\begin{DoxyCode}
1 catkin\_make run\_tests
\end{DoxyCode}


\subsection*{Building a map with Cartographer}

There are two bash scripts in the {\ttfamily scripts} folder which use \href{https://github.com/googlecartographer/cartographer_ros}{\tt Cartographer} to create a map of a racetrack. This map can then be used for different purposes, for example in the R\+OS navigation stack.


\begin{DoxyItemize}
\item To build a map while a roscore is running and providing sensor data, use the {\ttfamily cartographer\+\_\+online} script.
\item To build a map from a rosbag, use the {\ttfamily cartographer\+\_\+offline} script. The rosbag must provide range data on the rostopic {\ttfamily /scan} and a transformation tree on {\ttfamily /tf}; depending on your configuration of cartographer in {\ttfamily car\+\_\+cartographer/config} it may need to also have odometry data on {\ttfamily /odom} or I\+MU data on {\ttfamily /imu}.
\end{DoxyItemize}


\begin{DoxyCode}
1 # Either:
2 ./scripts/cartographer\_online.sh
3 # Or:
4 ./scripts/cartographer\_offline.sh /absolute/path/to/rosbag
\end{DoxyCode}


\subsection*{Documentation}


\begin{DoxyItemize}
\item For general information and documentation checkout the \href{https://github.com/Autonomous-Racing-PG/ros.package/wiki}{\tt wiki page}.
\item For source code documentation checkout the auto-\/generated \href{https://autonomous-racing-pg.github.io/ros.package/html/index.html}{\tt Doxygen documentation}.
\end{DoxyItemize}

\subsection*{License}

This project (exluded git submodules) is licensed under the M\+IT and G\+P\+Lv3 dual licensed -\/ see the \href{MIT.LICENSE}{\tt M\+I\+T.\+L\+I\+C\+E\+N\+SE} and \href{GPLv3.LICENSE}{\tt G\+P\+Lv3.\+L\+I\+C\+E\+N\+SE} file for details

\subsection*{Acknowledgments}


\begin{DoxyItemize}
\item TU Dortmund 
\end{DoxyItemize}